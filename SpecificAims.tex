%%%%% SPECIFIC AIMS %%%%%%

% 1 page maximum



%    Precise quantification of personal exposures to PM$_{2.5}$, PAHs, NO, and metals using advanced geospatial monitoring and modeling, overcoming limitations of previous studies relying on census tract pollution maps
%    Innovative integration of high-resolution exposure data with comprehensive individual health parameters to model the complex interplay between external exposures and internal susceptibility factors like obesity, diabetes, and inflammation
%    Development of the first comprehensive, multidisciplinary model elucidating the environmental drivers and biological mechanisms underlying NSCLC initiation and progression from air pollutants, by combining cutting-edge exposure science with epidemiology and molecular biology
%    Identification of causal mutational signatures linked to specific carcinogens under the mentorship of Dr. Paul Spellman, a renowned geneticist, using cutting-edge techniques to pinpoint molecular alterations induced by PM$_{2.5}$, PAHs, NO, and toxic metals, establishing a direct mechanistic link between environmental exposures and lung cancer development
%    Focus on environmental contributors in never-smokers, who often exhibit distinct molecular profiles, to investigate subtype-specific mechanisms and reveal novel vulnerabilities for targeted therapies
%    Leveraging advanced exposure monitoring integrated with epidemiological and molecular approaches to generate an unprecedented multidisciplinary model of environmental lung carcinogenesis, with the potential to transform our understanding of air pollution's role in NSCLC and identify new avenues for prevention and treatment, especially in high-risk never-smoker populations


%\paragraph{Precise Exposure Assessment}
%
%\paragraph{Integration of Exposures and Social Determinants}
%
%\paragraph{Multidisciplinary Carcinogenesis Model}
%
%\paragraph{Identification of Causal Mutational Signatures}
%
%\paragraph{Focus on Never-Smokers and Subtypes}
%
%\paragraph{Transformative Potential}
%

\newcommand{\SpecificAimOne}{Investigate the mechanistic link between PM$_{2.5}$ exposure and aggressive lung cancer biology in Black men and women}
\newcommand{\SpecificAimOneA}{Determine the association between PM$_{2.5}$ exposure and TP53 somatic mutations in a cohort of 505 lung cancer patients}
\newcommand{\SpecificAimOneB}{Assess the relationship between adverse social determinants of health and PM$_{2.5}$ exposure}
\newcommand{\SpecificAimOneC}{Validate the association between PM$_{2.5}$ and aggressive lung cancer biology using an independent cohort}
\newcommand{\SpecificAimOneD}{Integrate PM$_{2.5}$ exposure data with tumor genomic profiles to identify specific mutational signatures associated with air pollution exposure}

\newcommand{\SpecificAimTwo}{Characterize the timing and order of somatic mutations during lung cancer evolution}
\newcommand{\SpecificAimTwoA}{Reconstruct tumor phylogenetic trees to infer the timing of driver mutations and copy number alterations}
\newcommand{\SpecificAimTwoB}{Identify common patterns in the order of genomic events across lung cancer subtypes}
\newcommand{\SpecificAimTwoC}{Determine whether the timing of mutations differs between Black and White patients}

\newcommand{\SpecificAimThree}{Map cell-cell signaling networks in the lung tumor microenvironment}
\newcommand{\SpecificAimThreeA}{Reconstruct ligand-receptor interactions between malignant, immune, stromal and endothelial cells using single-cell RNA sequencing}
\newcommand{\SpecificAimThreeB}{Identify key signaling pathways and cell-cell interactions associated with early lung cancer development}
\newcommand{\SpecificAimThreeC}{Determine how the tumor microenvironment signaling network differs between Black and White patients}

\subsection{Specific Aims}

Lung cancer disproportionately affects Black men and women, who experience higher incidence and mortality rates compared to non-Hispanic Whites. 
Environmental exposures, such as air pollution, and tumor genomic profiles likely contribute to these disparities. 
The overarching goal of this research is to elucidate the molecular links between environmental exposures, social determinants, 
and tumor evolution Black lung cancer patients.

\paragraph*{Aim 1: \SpecificAimOne} \emph{Hypothesis: Exposure to PM$_{2.5}$ is associated with aggressive lung cancer biology in Black patients.}

\begin{itemize}[noitemsep]
	\item \SpecificAimOneA
	\item \SpecificAimOneB
	\item \SpecificAimOneC
	\item \SpecificAimOneD
\end{itemize}

\paragraph*{Aim 2: \SpecificAimTwo} \emph{Hypothesis: The timing and order of somatic mutations differs between Black and White lung cancer patients.}

\begin{itemize}[noitemsep]
	\item \SpecificAimTwoA
	\item \SpecificAimTwoB
	\item \SpecificAimTwoC
\end{itemize}

\paragraph*{Aim 3: \SpecificAimThree} \emph{Hypothesis: The lung tumor microenvironment signaling network is altered in Black patients compared to Whites.}

\begin{itemize}[noitemsep]
	\item \SpecificAimThreeA
	\item \SpecificAimThreeB
	\item \SpecificAimThreeC
\end{itemize}

This research will provide novel insights into the biological mechanisms underlying lung cancer disparities in Black patients. 
By integrating data on environmental exposures, tumor genomics, and the tumor microenvironment, 
we aim to identify specific molecular signatures and signaling pathways that contribute to aggressive lung cancer biology in this high-risk population. 
These findings could inform the development of targeted prevention and diagnosis strategies to reduce lung cancer disparities.


%%%%%%%%%%%%%

%\newcommand{\SpecificAimOne}{Start off strong with a major foray into safe work}
%\newcommand{\SpecificAimOneA}{A task that comes with convincing preliminary data}
%\newcommand{\SpecificAimOneB}{Another task that is essential to the later efforts}
%\newcommand{\SpecificAimOneC}{One more task that needs to be accomplished early on in the project}
%\newcommand{\SpecificAimOneD}{Validate model predictions of the relationship between signaling network state and resistance}
%
%\newcommand{\SpecificAimTwo}{That middle area where you will probably end up spending most of your time}
%\newcommand{\SpecificAimTwoA}{One of those tasks that just could not be skipped}
%\newcommand{\SpecificAimTwoB}{A task I am really looking forward to}
%\newcommand{\SpecificAimTwoC}{Something pulling this whole aim together}
%
%\newcommand{\SpecificAimThree}{A third major area that is quite risky}
%\newcommand{\SpecificAimThreeA}{Since we are just warming up this task is more likely to be feasible}
%\newcommand{\SpecificAimThreeB}{Getting this to happen will really be quite pricey}
%\newcommand{\SpecificAimThreeC}{This task really pulls everything together but will require everything working perfectly}
%
%
%\subsection{Specific Aims}
%
%Combination therapy holds considerable promise for overcoming intrinsic and acquired resistance to targeted therapies but relies on our ability to precisely identify the best drug combination for particular tumors. While immense focus exists on using genomic information to direct therapeutic approach, many resistance mechanisms can also arise from entirely tumor-extrinsic factors within the microenvironment. The \gls{rtk} AXL is widely implicated in resistance to targeted therapies such as those directed against EGFR. Regulation of AXL by \gls{ptdser}, as opposed to mutation, amplification or autocrine ligand, make identifying the tumors that will respond to AXL-targeted therapy especially challenging\cite{}.
%
%\paragraph*{Aim 1: \SpecificAimOne} \emph{Hypothesis: That Aim 1 is usually work with a greater chance of success.}
%
%\begin{itemize}[noitemsep]
%	\item \SpecificAimOneA
%	\item \SpecificAimOneB
%	\item \SpecificAimOneC
%	\item \SpecificAimOneD
%\end{itemize}
%
%\paragraph*{Aim 2: \SpecificAimTwo} \emph{Hypothesis: Middle Aims are where much of the real discovery occurs.}
%
%\begin{itemize}[noitemsep]
%	\item \SpecificAimTwoA
%	\item \SpecificAimTwoB
%	\item \SpecificAimTwoC
%\end{itemize}
%
%\paragraph*{Aim 3: \SpecificAimThree} \emph{Hypothesis: Third aims are less likely to be accomplished.}
%
%\begin{itemize}[noitemsep]
%	\item \SpecificAimThreeA
%	\item \SpecificAimThreeB
%	\item \SpecificAimThreeC
%\end{itemize}
%
%This work will considerably improve our ability to identify efficacious drug combinations by: (a) developing a mechanism-based assay for identifying which \glspl{rtk} are driving bypass resistance, (b) improving our basic understanding of exactly how network-level bypass resistance arises due to activation of non-targeted \glspl{rtk} both at the receptor-proximal and downstream signaling layer, and (c) expanding our understanding of the \gls{rtk} AXL with links to resistance, tumor spread, and immune avoidance.
