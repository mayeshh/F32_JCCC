%%%%% SPECIFIC AIMS %%%%%%

% 1 page maximum

\newcommand{\SpecificAimOne}{Start off strong with a major foray into safe work}
\newcommand{\SpecificAimOneA}{A task that comes with convincing preliminary data}
\newcommand{\SpecificAimOneB}{Another task that is essential to the later efforts}
\newcommand{\SpecificAimOneC}{One more task that needs to be accomplished early on in the project}
\newcommand{\SpecificAimOneD}{Validate model predictions of the relationship between signaling network state and resistance}

\newcommand{\SpecificAimTwo}{That middle area where you will probably end up spending most of your time}
\newcommand{\SpecificAimTwoA}{One of those tasks that just could not be skipped}
\newcommand{\SpecificAimTwoB}{A task I am really looking forward to}
\newcommand{\SpecificAimTwoC}{Something pulling this whole aim together}

\newcommand{\SpecificAimThree}{A third major area that is quite risky}
\newcommand{\SpecificAimThreeA}{Since we are just warming up this task is more likely to be feasible}
\newcommand{\SpecificAimThreeB}{Getting this to happen will really be quite pricey}
\newcommand{\SpecificAimThreeC}{This task really pulls everything together but will require everything working perfectly}


\subsection{Specific Aims}

Combination therapy holds considerable promise for overcoming intrinsic and acquired resistance to targeted therapies but relies on our ability to precisely identify the best drug combination for particular tumors. While immense focus exists on using genomic information to direct therapeutic approach, many resistance mechanisms can also arise from entirely tumor-extrinsic factors within the microenvironment. The \gls{rtk} AXL is widely implicated in resistance to targeted therapies such as those directed against EGFR. Regulation of AXL by \gls{ptdser}, as opposed to mutation, amplification or autocrine ligand, make identifying the tumors that will respond to AXL-targeted therapy especially challenging\cite{Meyer:CellSys}.

\paragraph*{Aim 1: \SpecificAimOne} \emph{Hypothesis: That Aim 1 is usually work with a greater chance of success.}

\begin{itemize}[noitemsep]
	\item \SpecificAimOneA
	\item \SpecificAimOneB
	\item \SpecificAimOneC
	\item \SpecificAimOneD
\end{itemize}

\paragraph*{Aim 2: \SpecificAimTwo} \emph{Hypothesis: Middle Aims are where much of the real discovery occurs.}

\begin{itemize}[noitemsep]
	\item \SpecificAimTwoA
	\item \SpecificAimTwoB
	\item \SpecificAimTwoC
\end{itemize}

\paragraph*{Aim 3: \SpecificAimThree} \emph{Hypothesis: Third aims are less likely to be accomplished.}

\begin{itemize}[noitemsep]
	\item \SpecificAimThreeA
	\item \SpecificAimThreeB
	\item \SpecificAimThreeC
\end{itemize}

This work will considerably improve our ability to identify efficacious drug combinations by: (a) developing a mechanism-based assay for identifying which \glspl{rtk} are driving bypass resistance, (b) improving our basic understanding of exactly how network-level bypass resistance arises due to activation of non-targeted \glspl{rtk} both at the receptor-proximal and downstream signaling layer, and (c) expanding our understanding of the \gls{rtk} AXL with links to resistance, tumor spread, and immune avoidance.
