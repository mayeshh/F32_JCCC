%%%%% SPECIFIC AIMS %%%%%%
% 1 page maximum%


\newcommand{\SpecificAimOne}{Develop a Hidden Markov Model using the PCAWG dataset to accurately time mutations in NSCLC}
\newcommand{\SpecificAimOneA}{Construct an initial HMM framework specifically tailored to capture the progression dynamics of NSCLC based on the PCAWG dataset}
\newcommand{\SpecificAimOneB}{Refine the model to incorporate single nucleotide variants (SNVs) from the PCAWG dataset as primary emissions for mutation timing}
\newcommand{\SpecificAimOneC}{Optimize the model's parameters, including transition probabilities and emission distributions, to enhance prediction accuracy}
\newcommand{\SpecificAimOneD}{Prepare the model framework for application on a broader dataset by ensuring scalability and adaptability}

\newcommand{\SpecificAimTwo}{Apply the refined HMM to the Black Lung Cancer Dataset to study mutation timing in NSCLC among African Americans}
\newcommand{\SpecificAimTwoA}{Adapt the HMM to accommodate genetic and mutational diversity specific to the African American NSCLC population}
\newcommand{\SpecificAimTwoB}{Analyze the mutation timings across different cities to identify common and unique mutational patterns}
\newcommand{\SpecificAimTwoC}{Correlate mutational timelines with available clinical outcomes to assess prognostic implications of mutation patterns}

\newcommand{\SpecificAimThree}{Validate the HMM's performance in predicting accurate mutation timings in NSCLC}
\newcommand{\SpecificAimThreeA}{Compare the HMM predictions with known clinical and pathological data to validate accuracy}
\newcommand{\SpecificAimThreeB}{Perform statistical validation using cross-validation methods to evaluate the model's robustness and reliability}
\newcommand{\SpecificAimThreeC}{Publish the validation results and methodology in a peer-reviewed computational biology journal}

\subsection{Specific Aims}

This proposal focuses on enhancing the accuracy of mutation timing in non-small cell lung cancer (NSCLC) through advanced computational modeling using a Hidden Markov Model (HMM). Our project leverages the extensive genomic data from the PCAWG dataset to develop a robust model that will be specifically applied to understand NSCLC progression in African Americans.

\paragraph*{Aim 1: \SpecificAimOne} \emph{Hypothesis: A specifically designed HMM can effectively use the PCAWG dataset to model mutation timing in NSCLC.}

\begin{itemize}[noitemsep]
    \item \SpecificAimOneA
    \item \SpecificAimOneB
    \item \SpecificAimOneC
    \item \SpecificAimOneD
\end{itemize}

\paragraph*{Aim 2: \SpecificAimTwo} \emph{Hypothesis: Applying the HMM to the Black Lung Cancer Dataset will reveal unique mutational timing and patterns specific to the African American population with NSCLC.}

\begin{itemize}[noitemsep]
    \item \SpecificAimTwoA
    \item \SpecificAimTwoB
    \item \SpecificAimTwoC
\end{itemize}

\paragraph*{Aim 3: \SpecificAimThree} \emph{Hypothesis: The HMM will provide accurate and clinically relevant predictions of mutation timings, enhancing our understanding of NSCLC progression in African Americans.}

\begin{itemize}[noitemsep]
    \item \SpecificAimThreeA
    \item \SpecificAimThreeB
    \item \SpecificAimThreeC
\end{itemize}

This computational approach aims to significantly advance our understanding of NSCLC progression by providing a precise temporal framework for mutation development, which could ultimately influence treatment strategies and outcomes for African American populations.



%%%%%%%%%%%%%%%%%%%%%%%%%%%%%%%

%% aim 1
%\newcommand{\SpecificAimOne}{Identify the Timing and Ordering of Driver Mutations and Copy Number Changes in NSCLC}
%% sub-aims
%\newcommand{\SpecificAimOneA}{Aggregate gene-level, pairwise timing estimates across many samples to identify the driver genes and CN gains in NSCLC}
%\newcommand{\SpecificAimOneB}{Infer sequences of genomic events, such as early driver events in NSCLC and late events like WGD} % is order correct?
%\newcommand{\SpecificAimOneC}{Extend the analysis to tumors not yet mapped, and identify the order of TP53 mutations/loss of chromosome 17, WGD, and CN changes} % redundant? need to define not yet mapped
%% possibly combine aims 1, 2, and 3?
%
%% aim 2
%\newcommand{\SpecificAimTwo}{Identify Changes in Mutational Signatures Between Early and Late Clonal Evolutionary Periods}
%% sub-aims
%\newcommand{\SpecificAimTwoA}{Catalog mutational signatures expressed in a three-base context centered on point mutations, identifying biases favoring mutation of one base pair given the neighboring bases}
%\newcommand{\SpecificAimTwoB}{Compare early clonal and late clonal SNVs to identify signatures most likely to decrease in activity (UV light, smoking) or increase (APOBEC) from early to late tumor evolution} %%% Maybe something about separating smoking effects from environmental effects?
%\newcommand{\SpecificAimTwoC}{Develop computational approaches to detect changes in mutational signature activity between early and late clonal periods} % extremely vague, also is this not already done?
%
%
%% aim 3
%\newcommand{\SpecificAimThree}{Model Molecular Clocks and Tumor Timing Using Sequential Tissue Samples} % am I understanding this right?
%% sub-aims
%\newcommand{\SpecificAimThreeA}{Utilize SBS1 (C to T in CpG) and SBS5 mutational processes as molecular clocks that reflect tissue age, with distinct rates in different tissues} % to identify the age of the tumor? what's the goal?
%\newcommand{\SpecificAimThreeB}{Model clock rates and rate increases for each tumor type using temporally separated tissue samples i.e. samples taken from the same patient over time} % is this too specific? are the taken from the same location or are they also spatially separated??
%\newcommand{\SpecificAimThreeC}{Apply the clock models to assess when tumors undergo major events like WGD and predict the timing of tumor development phases in NSCLC} 
%
%
%% aim 4
%\newcommand{\SpecificAimFour}{Generate Comprehensive Tumor Timing Maps for Each Cancer Type} 
%% sub-aims
%\newcommand{\SpecificAimFourA}{Combine timing results, ordering models, mutational signature changes, and molecular clock models to produce detailed timelines of tumor development for each cancer type} % summarize work in penultimate timing map
%% define what I mean by ``cancer type'', is this for NSCLC with differnet histological phenotypes?
%\newcommand{\SpecificAimFourB}{Apply the timing approaches to metastatic cancers in addition to primary tumors, demonstrating the ability to time tumor progression from pre-cancer to cancer and across metastases}
%\newcommand{\SpecificAimFourC}{Use the timing maps to identify common patterns and differences in the timing of driver mutations, WGD, and other key events across NSCLC heterogeneities} %?
%
%
%
%\subsection{Specific Aims} 
%
%This work will provide unprecedented insights into the timing and ordering of driver mutations, 
%copy number changes, and mutational processes during tumor evolution. 
%By modeling molecular clocks and generating comprehensive timing maps for tumors and metastases, 
%I will be able to predict the timing of key events in tumor development and progression, 
%including the transition from pre-cancer to cancer. 
%These timelines will allow us to correlate environmental exposure to pollutants with lung cancer occurrences in Blacks. 
%
%
%\paragraph*{Aim 1: \SpecificAimOne} \emph{Hypothesis: The ordering of driver mutations and copy number changes follows consistent patterns within NSCLC that can be inferred from aggregating timing estimates across many samples.} 
%
%\begin{itemize}[noitemsep]
%\item \SpecificAimOneA
%\item \SpecificAimOneB
%\item \SpecificAimOneC
%\end{itemize} 
%
%\paragraph*{Aim 2: \SpecificAimTwo} \emph{Hypothesis: Mutational signatures show characteristic changes in activity between early and late clonal evolutionary periods that can be detected computationally.} 
%
%\begin{itemize}[noitemsep]
%\item \SpecificAimTwoA
%\item \SpecificAimTwoB
%\item \SpecificAimTwoC
%\end{itemize} 
%
%\paragraph*{Aim 3: \SpecificAimThree} \emph{Hypothesis: Molecular clocks based on mutational processes can be used to model tumor timing and rate increases in a tissue and tumor-type specific manner.} 
%
%\begin{itemize}[noitemsep]
%\item \SpecificAimThreeA
%\item \SpecificAimThreeB
%\item \SpecificAimThreeC
%\end{itemize} 
%
%\paragraph*{Aim 4: \SpecificAimFour} \emph{Hypothesis: Integrating timing results, ordering models, mutational signature changes, and molecular clocks will enable the generation of comprehensive tumor timing maps that capture the full trajectory of tumor development.} 
%
%\begin{itemize}[noitemsep]
%\item \SpecificAimFourA
%\item \SpecificAimFourB
%\item \SpecificAimFourC
%\end{itemize}
%
%This work has the potential to significantly impact the scientific community by advancing our understanding 
%of environmental mutational signatures in NSCLC and developing tools to predict and prevent disease. 
%Importantly, the cohort is composed of Blacks in urban cities like LA, effectively promoting much needed diversity, equity, and inclusion in cancer research and care.
%This research will inform early detection strategies and guide the development of prevention and treatment approaches tailored to each stage of tumor evolution. 
%

%%%%%%%%%%%%%

%%%TEMPLATE%%%

%\newcommand{\SpecificAimOne}{Start off strong with a major foray into safe work}
%\newcommand{\SpecificAimOneA}{A task that comes with convincing preliminary data}
%\newcommand{\SpecificAimOneB}{Another task that is essential to the later efforts}
%\newcommand{\SpecificAimOneC}{One more task that needs to be accomplished early on in the project}
%\newcommand{\SpecificAimOneD}{Validate model predictions of the relationship between signaling network state and resistance}
%
%\newcommand{\SpecificAimTwo}{That middle area where you will probably end up spending most of your time}
%\newcommand{\SpecificAimTwoA}{One of those tasks that just could not be skipped}
%\newcommand{\SpecificAimTwoB}{A task I am really looking forward to}
%\newcommand{\SpecificAimTwoC}{Something pulling this whole aim together}
%
%\newcommand{\SpecificAimThree}{A third major area that is quite risky}
%\newcommand{\SpecificAimThreeA}{Since we are just warming up this task is more likely to be feasible}
%\newcommand{\SpecificAimThreeB}{Getting this to happen will really be quite pricey}
%\newcommand{\SpecificAimThreeC}{This task really pulls everything together but will require everything working perfectly}
%
%
%\subsection{Specific Aims}
%
%Combination therapy holds considerable promise for overcoming intrinsic and acquired resistance to targeted therapies but relies on our ability to precisely identify the best drug combination for particular tumors. While immense focus exists on using genomic information to direct therapeutic approach, many resistance mechanisms can also arise from entirely tumor-extrinsic factors within the microenvironment. The \gls{rtk} AXL is widely implicated in resistance to targeted therapies such as those directed against EGFR. Regulation of AXL by \gls{ptdser}, as opposed to mutation, amplification or autocrine ligand, make identifying the tumors that will respond to AXL-targeted therapy especially challenging\cite{}.
%
%\paragraph*{Aim 1: \SpecificAimOne} \emph{Hypothesis: That Aim 1 is usually work with a greater chance of success.}
%
%\begin{itemize}[noitemsep]
%	\item \SpecificAimOneA
%	\item \SpecificAimOneB
%	\item \SpecificAimOneC
%	\item \SpecificAimOneD
%\end{itemize}
%
%\paragraph*{Aim 2: \SpecificAimTwo} \emph{Hypothesis: Middle Aims are where much of the real discovery occurs.}
%
%\begin{itemize}[noitemsep]
%	\item \SpecificAimTwoA
%	\item \SpecificAimTwoB
%	\item \SpecificAimTwoC
%\end{itemize}
%
%\paragraph*{Aim 3: \SpecificAimThree} \emph{Hypothesis: Third aims are less likely to be accomplished.}
%
%\begin{itemize}[noitemsep]
%	\item \SpecificAimThreeA
%	\item \SpecificAimThreeB
%	\item \SpecificAimThreeC
%\end{itemize}
%
%This work will considerably improve our ability to identify efficacious drug combinations by: (a) developing a mechanism-based assay for identifying which \glspl{rtk} are driving bypass resistance, (b) improving our basic understanding of exactly how network-level bypass resistance arises due to activation of non-targeted \glspl{rtk} both at the receptor-proximal and downstream signaling layer, and (c) expanding our understanding of the \gls{rtk} AXL with links to resistance, tumor spread, and immune avoidance.
%