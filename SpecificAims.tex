%%%%% SPECIFIC AIMS %%%%%%
% 1 page maximum%

\newcommand{\SpecificAimOne}{Preprocess raw data from the Black Americans with NSCLC cohort}
\newcommand{\SpecificAimOneA}{Perform data cleaning and preprocessing by removing low-quality sequences, aligning the reads to the reference genome, and filtering data for further analysis with \texttt{GRITIC}}
\newcommand{\SpecificAimOneB}{Normalize the data to account for batch effects and other technical variances}
\newcommand{\SpecificAimOneC}{Detect and classify mutations (single nucleotide variants, insertions, deletions, etc.) using established bioinformatics pipelines to prepare the dataset for subsequent evolutionary analysis.}

\newcommand{\SpecificAimTwo}{Conduct evolutionary analysis to establish a baseline for subsequent comparison}
\newcommand{\SpecificAimTwoA}{Apply \texttt{GRITIC} and complementary tools from Gerstung et al., such as \texttt{cancerTiming}, \texttt{MutationTimeR}, and \texttt{PhylogicNDT}, to map the clonal and subclonal evolution of tumors in this dataset}
%\newcommand{\SpecificAimTwoB}{Integrate clinical data to correlate mutational patterns with treatment outcomes}
\newcommand{\SpecificAimTwoB}{Identify key mutation timings and evolutionary patterns and use as a baseline for further comparison and benchmarking}

\newcommand{\SpecificAimThree}{Develop innovative methods to map the paths of mutational processes throughout clonal evolution with improved mutation-time resolution}
\newcommand{\SpecificAimThreeA}{Establish high-resolution mutation-time techniques to precisely time mutational signature activities}
\newcommand{\SpecificAimThreeB}{Utilize the high-resolution timing data as inputs to improve the accuracy of timing key driver mutations}
\newcommand{\SpecificAimThreeC}{Develop accessible open-source software for use by the scientific community}


\subsection*{Specific Aims}

This proposal aims to identify unique mutational signatures, evolutionary patterns, 
and predictive biomarkers for treatment response and resistance in Black Americans 
with Non-Small Cell Lung Cancer (NSCLC) living in dense urban areas. 
Metastasis, the spread of cancer cells from the primary tumor to spatially separated sites, 
is the leading cause of cancer-related death. 
Understanding the evolutionary dynamics that lead to metastasis is vital for developing interventions 
to prevent and treat advanced stages of cancer. 
While significant progress has been made in understanding the early stages of tumor evolution, 
the genetic and evolutionary underpinnings of metastatic cancer remain less explored. 
This knowledge gap is particularly evident in the historically understudied Black American population.

\vspace{1em}
\noindent
By focusing on Black Americans, we seek to uncover molecular mechanisms of cancer that may be distinct 
from those observed in White individuals living in similar urban environments 
but not exposed to the same environmental and socioeconomic disadvantages. 
Black Americans often face higher exposure to air pollution and other environmental carcinogens due to systemic inequities in housing and urban planning. 
These factors, combined with unique genetic predispositions, may drive different mutational processes and cancer evolution trajectories.

\vspace{1em}
\noindent
This research is crucial for understanding the underlying mechanisms of NSCLC in Black Americans, 
which can lead to more personalized and effective treatment strategies. 
Furthermore, the findings from this study could inform public health policies aimed at reducing environmental risks 
and promoting equitable housing and environmental laws. 
Through innovative methodologies, we aim to narrow the gap in cancer research and enhance clinical outcomes for this underrepresented group.

\vspace{1em}
\noindent
This proposal builds on the pioneering work of leveraging whole genome sequencing to infer the evolutionary history of cancer. 
Advanced computational tools, such as the Gain Route Identification and Timing in Cancer (GRITIC) method, 
have been developed to time complex copy number gains and map the clonal evolution of tumors. 
These tools have provided critical insights into the subclonal architecture and evolutionary trajectories of various cancers, 
revealing pervasive intra-tumor heterogeneity and ordered paths of primary tumor evolution.

\vspace{1em}
\noindent
This project aims to revolutionize our understanding of NSCLC progression through cutting-edge advancements in modeling cancer evolution. 
By fulfilling these aims, the initiative is poised to bridge critical knowledge gaps in oncology, 
providing new analytical tools that map the evolution of cancer with unmatched precision. 
These advancements are expected to enhance the timing and specificity of treatment interventions 
and to lay the groundwork for tailored therapeutic strategies that are finely adjusted to the genetic and environmental profiles of individual patients, 
with special attention to the unique needs of Black communities. 
Ultimately, this research will be used to significantly improve early cancer detection, 
thereby transforming patient care and outcomes.

\vspace{1em}

\paragraph*{Aim 1: \SpecificAimOne} 
\emph{Hypothesis: Comprehensive data preprocessing will transform the raw data from the Black Americans with NSCLC cohort into a high-quality dataset suitable for detailed mutational and evolutionary analysis.}

\begin{itemize}[noitemsep]
    \item \SpecificAimOneA
    \item \SpecificAimOneB
    \item \SpecificAimOneC
\end{itemize}

\paragraph*{Aim 2: \SpecificAimTwo} 
\emph{Hypothesis: Applying advanced computational tools will reveal unique cancer evolutionary patterns, precise mutation timings, and distinct subclonal populations specific to the Black Americans with NSCLC cohort, providing a critical baseline for future studies.}

\begin{itemize}[noitemsep]
    \item \SpecificAimTwoA
    \item \SpecificAimTwoB
    %\item \SpecificAimTwoC
\end{itemize}

\paragraph*{Aim 3: \SpecificAimThree} 
\emph{Hypothesis: Improving mutation-time resolution of GRITIC will refine the accuracy of mapping mutational processes and timing driver mutations and enhance our understanding of cancer evolution.}

\begin{itemize}[noitemsep]
    \item \SpecificAimThreeA
    \item \SpecificAimThreeB
    \item \SpecificAimThreeC
\end{itemize}
