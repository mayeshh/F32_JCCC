\subsection{Significance}


% be sure to emphasize why the researchis relevant to the cancer problem. 
% If applicable, please also include how the research is relevant to UCLA JCCC’s catchment area, 
% Los Angeles County, and supports diversity, equity, and inclusion. 
% Also, discuss the feasibility of obtaining needed patient samples and include a biostatistical plan, if applicable.

%%%%%%%%%%%%%%%%%%%%%%%%%%%%%%%%%%%%%%%%%%


\paragraph{Lung Cancer and Environmental Exposures}
Lung cancer remains the leading cause of cancer-related mortality worldwide, with non-small cell lung cancer (NSCLC) 
accounting for 85\% of lung cancer cases in the US.~\cite{molina_nonsmall_2008} 
Akin to tobacco smoking, exposure to the complex mixture of air pollution, particularly fine particulate matter (PM$_{2.5}$) and nitric oxide (NO), 
poses a major risk factor for developing lung cancer. 
In heavily polluted cities like Los Angeles, exposure to these pollutants significantly increases the risk of developing lung cancer.~\cite{berg_air_2023,huang_air_2021} 
In 2014, the Nurse's Health Study found that living within 200 meters of a highway and a $10~\mu g/m^{3}$ increase in PM$_{2.5}$ 
levels were associated with an increased risk of lung cancer (HR = 1.57; 95\% CI: 1.26, 1.77).~\cite{puett_particulate_2014} 
Furthermore, a 2019 meta-analysis estimated that a $10~\mu g/m^{3}$ increase in PM$_{2.5}$ exposure in Europe and North America increased lung cancer risk by 25\%.~\cite{huang_ambient_2021}

\vspace{1em}
\noindent
Despite the clear evidence linking air pollution exposure to elevated lung cancer risk, 
the precise molecular mechanisms by which these complex pollutant mixtures initiate and promote NSCLC remain poorly understood, 
representing a critical knowledge gap. 
This study will investigate lung cancer in African Americans/Blacks (Black Americans), an understudied group that exhibits a high prevalence of aggressive, 
early-onset tumors that are often driven by distinct molecular profiles like EGFR mutations.~\cite{marcinkiewicz_impact_2023}
Elucidating the environmental drivers and biological pathways of lung carcinogenesis in this subgroup could reveal novel diagnostic approaches. 

\paragraph{Addressing Lung Cancer Inequities in Black Americans}
Although Black Americans have lower smoking rates compared to non-Hispanic Whites, 
they experience significantly higher lung cancer incidence and mortality rates, especially among men.~\cite{belony_abstract_2022,thaiparambil_abstract_2023,valencia_interrogating_2022} 
This disparity is striking, as Black Americans tend to initiate smoking later in life and consume fewer cigarettes compared to their White counterparts.~\cite{thaiparambil_abstract_2023,hede_drilling_2010} 
Black women, despite smoking fewer cigarettes, have the same or higher incidence of lung cancer as White women. %~\cite{} ???

\vspace{1em}
\noindent
Current lung cancer screening guidelines based on pack-years and age~\cite{landy_absolute_2024} fail to adequately identify Black Americans at risk.
Black Americans are diagnosed with lung cancer at a significantly younger age than Whites, often before reaching the screening threshold of 30 pack-years or age 55.~\cite{meza_evaluation_2021} 
The molecular drivers underlying these aggressive, early-onset lung cancers in the Black population remain unclear.
However, disparities in environmental exposures, particularly air pollution, are suspected to play a role.~\cite{marcinkiewicz_impact_2023}
Evidence shows that Black Americans are consistently exposed to significantly higher levels of PM$_{2.5}$ and NO compared to non-Hispanic Whites. %~\cite{hede_drilling_2010} ?
This study will utilize a multi-regional cohort of non-smokers, former smokers, and smokers,
to identify the molecular connections between air pollutants and lung cancer in Black Americans.

\vspace{1em}
\noindent
Furthermore, existing studies do not account for how social determinants of health in Black Americans may modulate susceptibility to cancers.~\cite{valencia_interrogating_2022} 
Addressing this gap is crucial for accurately assessing risk and developing prevention strategies in diverse populations.

\paragraph{Characterization of Environmental Exposure}
Outdoor air pollution, including PM$_{2.5}$, is classified as a Group 1 carcinogen by the International Agency for Research on Cancer (IARC).~\cite{gowda_ambient_2019} 
Past studies demonstrate a clear link between residing near major roadways and an elevated risk of developing lung cancer.~\cite{gowda_ambient_2019} 
Exhaust from combustion engines releases a mixture of carcinogenic compounds into the atmosphere near major roadways. 
These pollutants include polycyclic aromatic hydrocarbons (PAHs), nitrogen oxides, and toxic heavy metals such as arsenic, nickel, and lead.~\cite{yu_characterization_2015} 
Previous studies have attempted to map air pollution levels using census tract data.
However, these methods only detect a limited subset of pollutants, failing to capture the full complexity of environmental pollutants.
Moreover, existing research does not account for how rising global temperatures associated with climate change 
may alter the chemical composition and carcinogenic potency of air pollution over time. 
Another major shortcoming is the lack of integration of social determinants of health, 
such as obesity, diabetes, and chronic inflammatory conditions, which may exacerbate susceptibility to cancer.

\paragraph{Harnessing Advanced Computational Models for Precise Mutation Timing and Cancer Progression Analysis}

Gerstung \textit{et al.} apply a suite of sophisticated computational tools including 
\texttt{cancerTiming}, \texttt{MutationTimeR}, \texttt{PhylogicNDT SinglePatientTiming}, and \texttt{PhylogicNDT LeagueModel}, 
to analyze cancer progression and mutation timing. 
Another recent advancement in genomic analysis is \texttt{GRITIC}, detailed in Baker \textit{et al.}
This method utilizes advanced computational techniques to precisely time copy number gains in clonal populations. 
The strength of \texttt{GRITIC} lies in its ability to accurately determine the sequential timing of these copy number gains.

\vspace{1em}
\noindent
However, \texttt{GRITIC} and similar methods face limitations, such as the computational intensity of Bayesian inferences 
and Markov Chain Monte Carlo (MCMC) approaches, which restrict their application in large-scale or real-time scenarios. 
Additionally, they often require assumptions about mutation rates and copy number states that may not hold in all scenarios.
Currently, \texttt{GRITIC} requires DNA segments to have at least 20 single nucleotide variants (SNVs), 
and does not extend to systems with high copy number ($\geq~9$).~\cite{baker_history_2023}

\vspace{1em}
\noindent
Our proposed research aims to build upon and enhance the current \texttt{GRITIC} methodology by addressing these limitations. 
We will develop improved mutational signature activity timing approaches with higher resolution, 
enabling more precise tracking of mutagenic exposures over clonal evolutionary time. 
These refined timing methods will serve as priors for developing more accurate driver mutation timing approaches. 
By integrating temporally varying distributions of mutation signatures, 
we will significantly enhance the precision of mutation timing and the temporal ordering of key driver mutations.

%Our proposed research builds on the current \texttt{GRITIC} methodology by addressing these limitations and extending its capabilities. 
%First, we will develop mutational signature activity timing approaches with improved resolution. 
%This involves refining the timing of mutational signature activities, allowing us to track mutagenic exposures over clonal time with improved resolution. 
%The categorical approach of previous methods limited their resolution, and integrating our enhanced GRITIC model will improve these efforts. 
%We will calculate the proportion of SNVs at each clonal mutation multiplicity state within a gained segment to time individual copy number gains along mutation time.
%Next, we will leverage these refined mutational signature timing methods as priors to develop more accurate driver mutation timing approaches. 
%By integrating temporally varying distributions of the mutation signatures identified in the first aim, 
%we will extend GRITIC's capabilities to estimate the proportions of mutation signatures as they change over mutation time. 
%This joint estimation will improve the precision of both mutation timing and the temporal order of key driver mutations, 
%including SNVs that result in a gain or loss of function and oncogene amplifications.
%Additionally, we will develop a realistic statistical simulation framework to validate these novel approaches. 
%Using simulated, ground-truth tumor genomes with time-varied mutation signature proportions, 
%we will ensure our methods are robust and accurate. 
%These simulations will allow us to assess the performance of our enhanced models under various conditions and scenarios, 
%providing a comprehensive validation of their effectiveness.

\vspace{1em}
\noindent
Additionally, we are committed to creating open-source, user-friendly software tools to disseminate these advanced methodologies to the broader research community. 
By making these tools accessible, we aim to promote widespread adoption and foster collaborative efforts to advance cancer research.
Through these improvements, our research will overcome the current limitations of \texttt{GRITIC} and similar methods, 
providing more precise and scalable approaches to studying cancer evolution and the impact of environmental factors on mutational processes.

\paragraph{Potential for Transformative Impact}

This study will employ advanced geospatial methods to quantify individual exposures to air pollutants 
in Black communities in LA, Chicago, New Orleans, Charlestown SC, Richmond VA, and Rochester NY. 
Crucially, it will integrate this environmental exposure data with social determinants of health and biological factors 
that modulate disease susceptibility in these communities. 
Black populations in LA have historically faced disproportionately higher exposure to air pollution 
due to factors like redlining, the placing of industrial facilities near their neighborhoods, and a lack of green spaces. 
Despite having some of the lowest rates of smoking in the US, LA suffers from some of the worst highway-generated air pollution. 
By precisely characterizing these elevated exposures and combining them with data on 
obesity, diabetes, chronic inflammation, and other risk factors prevalent in Black communities, 
the goal is to develop a comprehensive analysis that elucidates how environmental drivers interact synergistically with 
social and biological parameters to initiate and promote aggressive, early-onset NSCLC in this population. 

\vspace{1em}
\noindent
This multidisciplinary approach, which combines external exposure assessments with internal susceptibility factors, 
is poised to provide novel mechanistic insights into the environmental carcinogenesis pathways 
that contribute to the excessive lung cancer burden observed in Black communities. 
By correlating precise air pollution exposure data with epidemiological cohorts and molecular tumor profiling from Black NSCLC patients, 
the research will forge a comprehensive model of how environmental toxins catalyze lung carcinogenesis 
amidst the backdrop of social and biological vulnerabilities in this underserved population. 
We anticipate that this innovative approach will provide new insights into the role of air pollution in 
the development of NSCLC among Black Americans. 
This will help us develop early detection strategies.
This is increasingly vital as, despite overall declining lung cancer rates, 
the incidence of NSCLC among women of color is rising in LA and similar urban areas.
