\subsection{Significance}


% be sure to emphasize why the researchis relevant to the cancer problem. 
% If applicable, please also include how the research is relevant to UCLA JCCC’s catchment area, 
% Los Angeles County, and supports diversity, equity, and inclusion. 
% Also, discuss the feasibility of obtaining needed patient samples and include a biostatistical plan, if applicable.

%%%%%%%%%%%%%%%%%%%%%%%%%%%%%%%%%%%%%%%%%%


\paragraph{Lung Cancer and Environmental Exposures}
Lung cancer remains the leading cause of cancer-related mortality worldwide, with non-small cell lung cancer (NSCLC) 
accounting for 85\% of lung cancer cases in the US.~\cite{molina_nonsmall_2008} 
Akin to tobacco smoking, exposure to the complex mixture of air pollution, particularly fine particulate matter (PM$_{2.5}$) and nitric oxide (NO), 
poses a major risk factor for developing lung cancer. 
In heavily polluted cities like Los Angeles, exposure to these pollutants significantly increases the risk of developing lung cancer.~\cite{x} 
In 2014, the Nurse's Health Study found that living within 200 meters of a highway and a $10~\mu g/m^{3}$ increase in PM$_{2.5}$ 
levels were associated with an increased risk of lung cancer (HR = 1.57; 95\% CI: 1.26, 1.77).~\cite{puett_particulate_2014} 
Furthermore, a 2019 meta-analysis estimated that a $10~\mu g/m^{3}$ increase in PM$_{2.5}$ exposure in Europe and North America increased lung cancer risk by 25\%.~\cite{huang_ambient_2021}

Despite the clear evidence linking air pollution exposure to elevated lung cancer risk, 
the precise molecular mechanisms by which these complex pollutant mixtures initiate and promote NSCLC remain poorly understood,~\cite{x} 
representing a critical knowledge gap. 
This study will investigate lung cancer in Blacks, an understudied group that exhibits a high prevalence of aggressive, 
early-onset tumors that are often driven by distinct molecular profiles like EGFR mutations.~\cite{x}
Elucidating the environmental drivers and biological pathways of lung carcinogenesis in this subgroup could reveal novel diagnostic approaches. 

\paragraph{Addressing Lung Cancer Inequities in Blacks}
Although African Americans/Blacks (Blacks) have lower smoking rates compared to non-Hispanic Whites, 
they experience significantly higher lung cancer incidence and mortality rates, especially among men.~\cite{x} 
This disparity is striking, as Blacks tend to initiate smoking later in life and consume fewer cigarettes compared to their White counterparts.~\cite{x} 
Black women, despite smoking fewer cigarettes, have the same or higher incidence of lung cancer as White women.~\cite{x}

Current lung cancer screening guidelines based on pack-years and age fail to adequately identify Blacks at risk.~\cite{x} 
Blacks are diagnosed with lung cancer at a significantly younger age than Whites, often before reaching the screening threshold of 30 pack-years or age 55.~\cite{x} 
The molecular drivers underlying these aggressive, early-onset lung cancers in the Black population remain unclear.~\cite{x} 
However, disparities in environmental exposures, particularly air pollution, are suspected to play a role.~\cite{x} 
Evidence shows that Blacks are consistently exposed to significantly higher levels of PM$_{2.5}$ and NO compared to non-Hispanic Whites.~\cite{x} 
This study will utilize a multi-regional cohort of non-smokers, former smokers, and smokers,
to identify the specific molecular connections between air pollutants and lung cancer in Blacks.

Furthermore, existing studies do not account for how social determinants of health may modulate susceptibility to environmentally induced cancers.~\cite{x} 
Addressing this gap is crucial for accurately assessing risk and developing prevention strategies in diverse populations.

\paragraph{Characterization of Environmental Exposure}
Outdoor air pollution, including PM$_{2.5}$, is classified as a Group 1 carcinogen by the International Agency for Research on Cancer (IARC).~\cite{x} 
Past studies demonstrate a clear link between residing near major roadways and an elevated risk of developing lung cancer.~\cite{x} 
Exhaust from combustion engines releases a cocktail of carcinogenic compounds into the atmosphere near major roadways. 
These pollutants include polycyclic aromatic hydrocarbons (PAHs), nitrogen oxides, and toxic heavy metals such as arsenic, nickel, and lead.~\cite{x} 
Previous studies have attempted to map air pollution levels using census tract data.~\cite{x} 
However, these methods only detect a limited subset of pollutants, failing to capture the full complexity of environmental pollutants.~\cite{x} 
Moreover, existing research does not account for how rising global temperatures associated with climate change 
may alter the chemical composition and carcinogenic potency of air pollution over time. 
Another major shortcoming is the lack of integration of social determinants of health, 
such as obesity, diabetes, and chronic inflammatory conditions, which may exacerbate susceptibility to cancer.

\paragraph{Potential for Transformative Impact}

This study will employ advanced geospatial methods to precisely quantify individual exposures to air pollutants 
in Black communities in LA, Chicago, New Orleans, Charlestown SC, Richmond VA, and Rochester NY. 
Crucially, it will integrate this environmental exposure data with social determinants of health and biological factors 
that modulate disease susceptibility in these communities. 
Black populations in LA have historically faced disproportionately higher exposure to air pollution 
due to factors like redlining, the placing of industrial facilities near their neighborhoods, and a lack of green spaces. 
Despite having some of the lowest rates of smoking in the US, LA suffers from some of the worst highway-generated air pollution. 
By precisely characterizing these elevated exposures and combining them with data on 
obesity, diabetes, chronic inflammation, and other risk factors prevalent in Black communities, 
the goal is to develop a comprehensive model elucidating how environmental drivers interact synergistically with 
social and biological parameters to initiate and promote aggressive, early-onset NSCLC in this population. 

This multidisciplinary approach, combining external exposure assessments with internal susceptibility factors, 
will provide novel mechanistic insights into the environmental carcinogenesis pathways driving the excess lung cancer burden observed in Black communities. 
By integrating precise air pollution exposure data with epidemiological cohorts and molecular tumor profiling from Black NSCLC patients, 
this study will generate a comprehensive model of how environmental insults precipitate lung carcinogenesis 
in the context of social and biological vulnerabilities in this underserved population. 
Insights from this innovative approach have the potential to transform our understanding of air pollution's role in NSCLC etiology in Blacks 
and identify new opportunities for targeted prevention, early detection, and treatment strategies.
This is particularly important as while the rates of most lung cancers are declining, 
the incidence of NSCLC in non-smoking women of color is rapidly rising in LA and other cities.~\cite{x}

%\paragraph{Air Pollution and Lung Cancer Risk in LA County}
%
%
%Akin to tobacco smoking, exposure to the complex mixture of air pollution, 
%particularly fine particulate matter (PM$_{2.5}$) and nitric oxide (NO), 
%poses a major risk factor for developing lung cancer.
%In heavily polluted cities like Los Angeles, exposure to these pollutants significantly increases the risk of developing lung cancer. 
%In 2014, the Nurse's Health Study found that living within 200 meters of a highway and a 10 $\mu g/m^{3}$ increase in PM$_{2.5}$ levels 
%were associated with an increased risk of lung cancer (HR = 1.57; 95\% CI: 1.26, 1.77)~\cite{x}. 
%Furthermore, a 2019 meta-analysis estimated that a 10 $\mu g/m^{3}$ increase in PM$_{2.5}$ exposure in Europe and North America increased lung cancer risk by 25\%~\cite{x}.
%
%
%\cbstart % adds a vertical line on left side of text
%\lipsum[3-3]
%\cbend
%
%\paragraph{Addressing Lung Cancer Inequities in Blacks}
%
%
%Although African Americans/Blacks (Blacks) have lower smoking rates compared to non-Hispanic Whites, 
%they experience significantly higher lung cancer incidence and mortality rates, especially among men.
%This disparity is striking, as Blacks tend to initiate smoking later in life and consume fewer cigarettes 
%compared to their White counterparts. 
%
% AA/Black women, again, despite smoking fewer cigarettes, have the same or higher incidence of lung cancer as White women. 
%Current lung cancer screening guidelines based on pack-years and age fail to adequately identify Blacks at risk.
%Blacks are diagnosed with lung cancer at a significantly younger age than Whites, 
%often before reaching the screening threshold of 30 pack-years or age 55~\cite{x}. 
%The molecular drivers underlying these aggressive, early-onset lung cancers in the Black population remain unclear. 
%However, disparities in environmental exposures, particularly air pollution, are suspected to play a role~\cite{x}. 
%Evidence shows that Blacks are consistently exposed to significantly higher levels of PM$_{2.5}$ and NO compared to non-Hispanic Whites~\cite{x}. % very similar to the original
%This study will utilize a multi-regional cohort to identify the specific molecular connections 
%between air pollutants and lung cancer in Blacks, regardless of their smoking status.
%
%The exposure to elevated levels of PM$_{2.5}$ and NO can be attributed to historical practices of segregation, redlining, 
%and the deliberate placement of highways, industrial buildings, and waste facilities in predominantly Black neighborhoods~\cite{x}. 
%This has the effects of relegating Blacks to neighborhoods without trees and an abundance of heat-trapping pavement~\cite{x}.
%These inequalities in the environment are directly reflected in the observed lung cancer disparities between Blacks and Whites.
%
%\paragraph{Characterization of Environmental Exposure}
%
%
%Outdoor air pollution, including PM$_{2.5}$, is classified as a Group 1 carcinogen by the International Agency for Research on Cancer (IARC)~\cite{x}. 
%Past studies demonstrate a clear link between residing near major roadways and an elevated risk of developing lung cancer~\cite{x}. 
%However, the precise molecular pathways through which air pollutants initiate and promote the progression of lung malignancies remain poorly understood~\cite{x}.
%
%Exhaust from combustion engines releases a cocktail of carcinogenic compounds into the atmosphere near major roadways. 
%These pollutants include polycyclic aromatic hydrocarbons (PAHs), nitrogen oxides, and toxic heavy metals such as arsenic, nickel, and lead.~\cite{x} 
%Previous studies have attempted to map air pollution levels using census tract data.
%However, these methods only detect a limited subset of pollutants, failing to capture the full complexity of roadway emissions. 
%Moreover, existing research does not account for how rising global temperatures associated with climate change 
%may alter the chemical composition and carcinogenic potency of air pollution over time. 
%Another major shortcoming of these studies is the lack of integration of social determinants of health, such as obesity, diabetes, 
%and chronic inflammatory conditions, which may exacerbate susceptibility to cancer. 
%Accounting for these factors is crucial for accurately assessing cancer risk from exposure to roadway air pollution.
%
%This study will employ advanced geospatial methods to precisely quantify individual exposures to air pollutants. 
%Crucially, it will integrate environmental exposure data with social and biological factors that modulate disease susceptibility. 
%The goal is to develop a comprehensive model elucidating how environmental drivers interact with human health parameters 
%to initiate and promote non-small cell lung cancer (NSCLC) development and progression. 
%This multidisciplinary approach, combining external exposure assessment with internal risk factors, 
%will provide novel mechanistic insights into the environmental carcinogenesis of lung cancer.
%
% 
%\paragraph{Identifying Molecular Drivers of Lung Carcinogenesis}
% 
% 
%The vast majority of environmental cohort studies identify important associations but are limited in their ability to determine molecular causation. 
%Air pollution contains complex elements, and individuals often have complex exposures to pollution, and secondhand smoke, or may have lived in multiple locations. 
%Consequently, association studies are becoming increasingly inadequate. Recent advances have made it possible 
%to identify distinct causal molecular signatures within a tumor and pinpoint the temporal evolution of these signatures during carcinogenesis. 
%Specific molecular signatures have been identified that result from specific environmental exposures or mutagens, ranging from aromatic amines and tobacco smoke to heavy metals. 
%Yu and colleagues found the mutation spectrum of air pollution-related lung cancers and provided evidence of an air pollution exposure-genomic mutation relationship on a large scale~\cite{x}. 
%Whole-genome sequencing (WGS) of human tumors has revealed distinct mutation patterns linked to specific exposures. 
%These mutational signatures permit further exploration of the roles of environmental agents in cancer etiology and underscore how DNA is directly vulnerable to environmental agents. 
%Furthermore, we can use these tools to measure when during tumor development these exposures exerted their mutagenic effects~\cite{x}. 
%In this study, we aim to use these molecular tools to precisely identify causal environmental exposures that promote initiation, progression, and aggressive NSCLC biology.
% 
%%%%%%%%%%%%%%%%
% 
%%% copy-paste from grant - placeholder, need to rewrite
%
%The vast majority of environmental cohort studies, identify important associations but  are limited because they cannot determine molecular causation. 
%Air pollution contains complex elements; individuals frequently have complex exposures to pollution and second-hand smoke or may have lived in multiple locations. 
%Consequently, association studies are becoming increasingly inadequate. 
%Recently, it has become possible to identify distinct causal molecular signatures within a tumor and pinpoint the temporal evolution of these signatures during carcinogenesis. 
%Specific molecular signatures have been identified that result from specific environmental exposures or mutagens, ranging from aromatic amines, and tobacco smoke, to heavy metals. 
%Yu and colleaguesR found the mutation spectrum of air pollution-related lung cancers and provided evidence of an air pollution exposure–genomic mutation relationship on a large scale. 
%Whole-genome sequencing (WGS) of human tumors has revealed distinct mutation patterns linked to specific exposures. 
%These mutational signatures permit further exploration of the roles of environmental agents in cancer etiology and underscores how DNA is directly vulnerable to environmental agents. 
%Further, we can use these tools to measure when during tumor development these exposures exerted their mutagenic effects (Gerstung 2020). 
%We aim to use these molecular tools to precisely identify causal environmental exposures that promote initiation, progression, and aggressive NSCLC biology. Current lung cancer screening guidelines based on pack-years and age fail to adequately identify African Americans/Blacks at riay~\cite{x