\section{Significance}


% be sure to emphasize why the researchis relevant to the cancer problem. 
% If applicable, please also include how the research is relevant to UCLA JCCC’s catchment area, 
% Los Angeles County, and supports diversity, equity, and inclusion. 
% Also, discuss the feasibility of obtaining needed patient samples and include a biostatistical plan, if applicable.

%%%%%%%%%%%%%%%%%%%%%%%%%%%%%%%%%%%%%%%%%%


\paragraph{Environmental risk factors in lung cancer}

%%% copy-paste from grant - placeholder, need to rewrite

% - define PM$_{2.5}$
% - get references

While tobacco is an important risk factor for lung cancer, air pollution is also a major contributor, 
particularly in heavily polluted cities like Los Angeles and Chicago. 
Air pollution is a complex mixture, containing PM$_{2.5}$ and nitric oxide (NO) – both major carcinogens~\cite{}. 
In both smokers and non-smokers, the Nurse’s Health Study found in 2014, that proximity to highways (<200 meters) 
and a 10 μg/m3 increase in PM$_{2.5}$ , significantly increased lung cancer-risk (HR = 1.57; 95\% CI: 1.26, 1.77)~\cite{}.  
More recently, a 2019 meta-analysis of PM$_{2.5}$ exposure in Europe and North American estimated that a 10 ug/m3 increase in PM$_{2.5}$ increased lung cancer risk by 25\%~\cite{}.

%\cbstart % adds a vertical line on left side of text
%\lipsum[3-3]
%\cbend

\paragraph{Lung cancer disparities in AA/Blacks}

\lipsum[4-5]

\paragraph{Quantitative characterization of exposures}

\lipsum[6-10]
 
 \paragraph{Identifying molecular causations}