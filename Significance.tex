\subsection{Significance}


% be sure to emphasize why the researchis relevant to the cancer problem. 
% If applicable, please also include how the research is relevant to UCLA JCCC’s catchment area, 
% Los Angeles County, and supports diversity, equity, and inclusion. 
% Also, discuss the feasibility of obtaining needed patient samples and include a biostatistical plan, if applicable.

%%%%%%%%%%%%%%%%%%%%%%%%%%%%%%%%%%%%%%%%%%


\paragraph{Lung Cancer and Environmental Exposures}
Lung cancer remains the leading cause of cancer-related mortality worldwide, with non-small cell lung cancer (NSCLC) 
accounting for 85\% of lung cancer cases in the US.~\cite{molina_nonsmall_2008} 
Akin to tobacco smoking, exposure to the complex mixture of air pollution, particularly fine particulate matter (PM$_{2.5}$) and nitric oxide (NO), 
poses a major risk factor for developing lung cancer. 
In heavily polluted cities like Los Angeles, exposure to these pollutants significantly increases the risk of developing lung cancer.~\cite{berg_air_2023,huang_air_2021} 
In 2014, the Nurse's Health Study found that living within 200 meters of a highway and a $10~\mu g/m^{3}$ increase in PM$_{2.5}$ 
levels were associated with an increased risk of lung cancer (HR = 1.57; 95\% CI: 1.26, 1.77).~\cite{puett_particulate_2014} 
Furthermore, a 2019 meta-analysis estimated that a $10~\mu g/m^{3}$ increase in PM$_{2.5}$ exposure in Europe and North America increased lung cancer risk by 25\%.~\cite{huang_ambient_2021}

Despite the clear evidence linking air pollution exposure to elevated lung cancer risk, 
the precise molecular mechanisms by which these complex pollutant mixtures initiate and promote NSCLC remain poorly understood, 
representing a critical knowledge gap. 
This study will investigate lung cancer in African Americans/Blacks (Black Americans), an understudied group that exhibits a high prevalence of aggressive, 
early-onset tumors that are often driven by distinct molecular profiles like EGFR mutations.~\cite{marcinkiewicz_impact_2023}
Elucidating the environmental drivers and biological pathways of lung carcinogenesis in this subgroup could reveal novel diagnostic approaches. 

\paragraph{Addressing Lung Cancer Inequities in Black Americans}
Although Black Americans have lower smoking rates compared to non-Hispanic Whites, 
they experience significantly higher lung cancer incidence and mortality rates, especially among men.~\cite{belony_abstract_2022,thaiparambil_abstract_2023,valencia_interrogating_2022} 
This disparity is striking, as Black Americans tend to initiate smoking later in life and consume fewer cigarettes compared to their White counterparts.~\cite{thaiparambil_abstract_2023,hede_drilling_2010} 
Black women, despite smoking fewer cigarettes, have the same or higher incidence of lung cancer as White women. %~\cite{} ???

Current lung cancer screening guidelines based on pack-years and age~\cite{landy_absolute_2024} fail to adequately identify Black Americans at risk.
Black Americans are diagnosed with lung cancer at a significantly younger age than Whites, often before reaching the screening threshold of 30 pack-years or age 55.~\cite{meza_evaluation_2021} 
The molecular drivers underlying these aggressive, early-onset lung cancers in the Black population remain unclear.
However, disparities in environmental exposures, particularly air pollution, are suspected to play a role.~\cite{marcinkiewicz_impact_2023}
Evidence shows that Black Americans are consistently exposed to significantly higher levels of PM$_{2.5}$ and NO compared to non-Hispanic Whites. %~\cite{hede_drilling_2010} ?
This study will utilize a multi-regional cohort of non-smokers, former smokers, and smokers,
to identify the molecular connections between air pollutants and lung cancer in Black Americans.

Furthermore, existing studies do not account for how social determinants of health in Black Americans may modulate susceptibility to cancers.~\cite{valencia_interrogating_2022} 
Addressing this gap is crucial for accurately assessing risk and developing prevention strategies in diverse populations.

\paragraph{Characterization of Environmental Exposure}
Outdoor air pollution, including PM$_{2.5}$, is classified as a Group 1 carcinogen by the International Agency for Research on Cancer (IARC).~\cite{gowda_ambient_2019} 
Past studies demonstrate a clear link between residing near major roadways and an elevated risk of developing lung cancer.~\cite{gowda_ambient_2019} 
Exhaust from combustion engines releases a mixture of carcinogenic compounds into the atmosphere near major roadways. 
These pollutants include polycyclic aromatic hydrocarbons (PAHs), nitrogen oxides, and toxic heavy metals such as arsenic, nickel, and lead.~\cite{yu_characterization_2015} 
Previous studies have attempted to map air pollution levels using census tract data.
However, these methods only detect a limited subset of pollutants, failing to capture the full complexity of environmental pollutants.
Moreover, existing research does not account for how rising global temperatures associated with climate change 
may alter the chemical composition and carcinogenic potency of air pollution over time. 
Another major shortcoming is the lack of integration of social determinants of health, 
such as obesity, diabetes, and chronic inflammatory conditions, which may exacerbate susceptibility to cancer.

\paragraph{Harnessing Advanced Computational Models for Precise Mutation Timing and Cancer Progression Analysis}

% Detailed Methods from Gerstung et al.
Gerstung \textit{et al.}, applies a suite of sophisticated computational tools including 
cancerTiming, MutationTimeR, PhylogicNDT SinglePatientTiming, and PhylogicNDT LeagueModel, 
to analyze cancer progression and mutation timing. 
Another recent advancement in genomic analysis is the Gain Route Identification and Timing In Cancer (GRITIC) method, 
detailed in Baker \textit{et al.}
This method employs advanced computational techniques to analyze complex genetic variations 
and is particularly adept at handling large-scale genomic datasets. 
The strength of GRITIC lies in its ability to time sequential copy number gains with high accuracy.
However, the sophisticated Markov Chain Monte Carlo (MCMC) approach becomes computationally intractable 
for large copy numbers states ($\geq 9$).~\cite{baker_history_2023}

These methods utilize advanced probabilistic and statistical techniques to analyze mutational landscapes 
and identify evolutionary patterns in cancer genomes.
However, these complex Bayesian inferences and MCMC approaches can be computationally intensive, 
limiting their use in large-scale or real-time scenarios.
In addition, tools such as CancerTiming, MutationTimeR, and GRITIC focus on estimating the timing of clonal chromosomal gains and mutations 
but often require assumptions about mutation rates and copy number states that may not hold in all scenarios. 

% Explain HMM
Hidden Markov Models (HMMs) are useful for modeling time-series data where the states of the system are hidden 
and must be inferred through observable events. 
HMMs capture transitions between hidden states based on observed data, 
making them highly effective for sequential prediction tasks such as understanding disease progression.

% Explain what an LSTM is 
Long Short-Term Memory (LSTM) networks are a specialized type of neural network that are 
highly effective in processing and retaining information across lengthy data sequences.
This feature is particularly well-suited for modeling the complex evolutionary trajectories that occur in cancer. 
This allows the LSTM to dynamically learn from evolving genetic sequences and adapt to new patterns as they emerge.

% Why the LSTM-HMM approach is better than current methods
Combining an LSTM with an HMM will enable dynamic modeling capabilities from LSTMs with the probabilistic modeling strengths of HMMs, 
with the potential to significantly improve cancer evolution studies. 
LSTMs provide a deeper and more nuanced understanding of long-term dependencies and non-linear interactions in genomic sequences, 
offering transformative improvements in the timing and ordering of mutations during cancer progression.
When integrated with HMMs, which effectively model hidden states and transition probabilities, 
this approach allows for a more dynamic analysis that can adapt to new data, enhance prediction accuracy, 
and uncover hidden evolutionary pathways in cancer progression. 
This makes the LSTM-HMM model particularly powerful for predicting disease trajectories and improving treatment strategies, 
offering a superior alternative to traditional and current computational methods used in cancer genomics.

\paragraph{Potential for Transformative Impact}

This study will employ advanced geospatial methods to quantify individual exposures to air pollutants 
in Black communities in LA, Chicago, New Orleans, Charlestown SC, Richmond VA, and Rochester NY. 
Crucially, it will integrate this environmental exposure data with social determinants of health and biological factors 
that modulate disease susceptibility in these communities. 
Black populations in LA have historically faced disproportionately higher exposure to air pollution 
due to factors like redlining, the placing of industrial facilities near their neighborhoods, and a lack of green spaces. 
Despite having some of the lowest rates of smoking in the US, LA suffers from some of the worst highway-generated air pollution. 
By precisely characterizing these elevated exposures and combining them with data on 
obesity, diabetes, chronic inflammation, and other risk factors prevalent in Black communities, 
the goal is to develop a comprehensive analysis that elucidates how environmental drivers interact synergistically with 
social and biological parameters to initiate and promote aggressive, early-onset NSCLC in this population. 

Integrating these assessments with the advanced capabilities of a hybrid HMM-LSTM model, 
the study will enhance the precision in mapping the timing of mutational events and uncover complex patterns of NSCLC progression. 
This multidisciplinary approach, which combines external exposure assessments with internal susceptibility factors, 
is poised to provide novel mechanistic insights into the environmental carcinogenesis pathways 
that contribute to the excessive lung cancer burden observed in Black communities. 
By correlating precise air pollution exposure data with epidemiological cohorts and molecular tumor profiling from Black NSCLC patients, 
the research will forge a comprehensive model of how environmental toxins catalyze lung carcinogenesis 
amidst the backdrop of social and biological vulnerabilities in this underserved population. 
We anticipate that this innovative approach will provide new insights into the role of air pollution in 
the development of NSCLC among Black Americans. 
This will help us develop targeted prevention, early detection, and treatment strategies.
This is increasingly vital as, despite overall declining lung cancer rates, 
the incidence of NSCLC among women of color is rising in LA and similar urban areas. %cite?

