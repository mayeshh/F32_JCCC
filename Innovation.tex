\subsection{Innovation}

This proposal introduces several innovative elements to redefine our understanding of the 
role of air pollution in lung cancer in Black Americans. 
Traditional studies have often relied on broad regional data like census tract pollution maps 
that do not adequately capture individual exposures or the complex nature of environmental pollutants. 
In contrast, this study utilizes advanced geospatial monitoring to quantify personal exposures to 
PM$_{2.5}$, PAHs, NO, and metals, which are crucial for establishing clear exposure-response relationships.

A novel aspect of this proposal is the integration of an LSTM layer with an HMM.
This hybrid approach is expected to enhance the precision of timing mutational events in NSCLC. 
The LSTM component is adept at analyzing long sequences of data, 
capturing temporal dependencies and patterns that may be indicative of mutational triggers linked to environmental factors. 
This capability is combined with the probabilistic power of HMMs, 
which model the hidden states and transition probabilities, offering a detailed temporal analysis of mutational sequences. 
Together, they provide a comprehensive view of how mutations develop and progress in response to specific environmental exposures.

Under the mentorship of Dr. Paul Spellman, a leader in identifying mutational signatures linked to specific carcinogens, 
this research will employ cutting-edge genomic techniques to meticulously chart the molecular alterations induced by environmental exposures. 
The LSTM-HMM allows for a dynamic and nuanced analysis, 
setting this approach apart by providing a robust framework for directly connecting 
specific components of air pollution to the pathogenesis of lung cancer in Black Americans. 
The hybrid model will rigorously define the temporal and causal relationships between environmental 
exposures and mutational events, establishing a direct and unambiguous link between air pollution and lung cancer development.

Overall, this proposal leverages sophisticated exposure monitoring 
integrated with advanced computational modeling to better understand environmental lung carcinogenesis. 
This multi-disciplinary approach, combining high-resolution environmental data with innovative computational techniques, 
has the potential to transform our understanding of the role of air pollution in NSCLC 
and identify new avenues for prevention and early detection in Black Americans.


%%%%%%%%%%%%%%%

%This proposal introduces several innovative elements to redefine our understanding of the role of air pollution in lung cancer in Black Americans. 
%Traditional studies have often relied on broad regional data like census tract pollution maps 
%that do not adequately capture individual exposures or the complex nature of environmental pollutants. 
%In contrast, this study utilizes advanced geospatial monitoring to precisely quantify personal exposures to 
%PM$_{2.5}$, PAHs, NO, and metals, which are crucial for establishing clear exposure-response relationships.
%
%A novel aspect of this proposal is the use of a Hidden Markov Model (HMM) to enhance the precision in mapping the timing of mutational events in NSCLC. 
%This computational approach allows for a detailed temporal analysis of mutational sequences, 
%offering insights into when specific mutations triggered by environmental carcinogens occur during cancer evolution. 
%By integrating this model with high-resolution exposure data and comprehensive individual health parameters, 
%we can dissect the complex interplay between external environmental exposures and internal genetic susceptibility.
%
%Under the mentorship of Dr. Paul Spellman, a leader in identifying mutational signatures linked to specific carcinogens, 
%this research will employ cutting-edge genomic techniques to meticulously chart the molecular alterations induced by environmental exposures. 
%This approach is set to provide a robust framework for connecting specific components of air pollution directly to the pathogenesis of lung cancer in Black Americans. 
%The proposed HMM will rigorously define the temporal and causal relationships between environmental exposures and mutational events, 
%establishing a direct and unambiguous link between air pollution and lung cancer development.
%
%Overall, this proposal leverages sophisticated exposure monitoring integrated with 
%advanced computational modeling to better understand environmental lung carcinogenesis. 
%This multi-disciplinary approach has the potential to transform our understanding of the role of air pollution in NSCLC 
%and identify new avenues for prevention and early detection, especially in high-risk never-smoker populations.


%%%%%%%%%%%%%%%%%%%%%%%%%%%

%While lung cancer in smokers is well-studied, far less is known about environmental contributors in Black non-smokers. 
%Investigating subtype-specific mechanisms could reveal novel vulnerabilities for targeted therapies. % not sure where this sentance should go
%This proposal introduces several innovative elements to advance our understanding of air pollution's role in lung cancer. 
%While previous studies have relied on census tract pollution maps that fail to capture individual exposures and pollutant complexity, 
%this study pioneers advanced geospatial monitoring and modeling to precisely quantify personal exposures to PM$_{2.5}$, PAHs, NO, 
%and metals — critical for defining exposure-response relationships. 
%Existing studies have not accounted for how social determinants like obesity, diabetes, and inflammation modulate environmental cancer risk. 
%This proposal uniquely integrates high-resolution exposure data with comprehensive individual health parameters 
%to model the complex interplay between external exposures and internal susceptibility. 
%By combining exposure science with epidemiology and molecular biology, 
%this study will provide a comprehensive model that elucidates the environmental drivers and biological mechanisms underlying 
%environmentally induced NSCLC initiation and progression. 
%
%Under the mentorship of Dr. Paul Spellman, I will leverage his scientific expertise to identify causal mutational signatures linked to specific carcinogens. 
%I will employ the cutting-edge techniques developed by Dr. Spellman to meticulously pinpoint the molecular alterations induced by environmental exposures, 
%thereby providing a direct link between the components of air pollution and the pathogenesis of lung cancer. 
%Through the implementation of the proposed research, I aim to elucidate the precise mutational signatures associated with exposure to 
%PM$_{2.5}$, PAHs, NO, and toxic metals found within the complex mixture of air pollution. 
%By rigorously identifying these causal relationships, I will establish a direct and unambiguous 
%connection between environmental exposures and the development of lung cancer.
%
%In summary, this proposal leverages advanced exposure monitoring integrated with epidemiological and molecular approaches 
%to generate an unprecedented multidisciplinary model of environmental lung carcinogenesis. 
%This multi-disciplinary approach has the potential to transform our understanding of the role of air pollution in NSCLC 
%and identify new avenues for prevention and early detection, especially in high-risk never-smoker populations.
%
%%%%%%%%%%%%%%

%Our team has a proven track record of recruiting diverse individuals to clinical trials: 
%Our team brings together leaders in the field of cancer health disparities – 
%Rob Winn, Victoria Seewaldt, Rick Kittles, Augusto Ochoa, Chanita Hughes Halbert -  
%who work with communities to recruit AA/Black to clinical trials – with the goals of improving early detection and screening and reducing cancer mortality.

% We have the scientific capacity to identify causal mutational signatures: 
% Our basic scientists are leaders in identifying mutational signatures caused by specific carcinogens (Paul Spellman) 
%and cell-cell signaling networks within the lung pre-cancerous and cancer microenvironment (Sylvia Plevritis). 

% Our team collects tissue from cities such as Los Angeles: Los Angeles the lowest smoking rates in the United States but unfortunately, some of our worst air pollution. 
%Los Angeles has a rapidly rising rate of non-smoking related adenocarcinomas of the lung. 
%Because our smoking rates are so low, and our pollution so bad, we can identify mutational signatures in these cancers and correlate them with specific exposures. 

% We have the geospatial tools and expertise to go beyond census maps and pinpoint specific exposures: 
%We are able to precisely evaluate exposures for individuals using wearable devices (Marta Jankowska). 
%We have the capacity to integrate exposure data with satellite traffic flow and global atmospheric data and global warming (Dan Crichton, Ashish Mahabal). 
%Our team has the capacity to perform multi-scale analysis and integrating the impact of environmental exposures and biological signatures (Terry Hyslop). 
